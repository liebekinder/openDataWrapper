%%This is a very basic article template.
%%There is just one section and two subsections.
\documentclass[a4paper]{article}

\usepackage[utf8]{inputenc}
\usepackage[T1]{fontenc}
\usepackage[francais]{babel}
\usepackage[colorlinks=true]{hyperref}
\usepackage{verbatim} % adds environment for commenting out blocks of text & for
% better verbatim
\usepackage{xcolor}
\usepackage{listings}
\usepackage{caption}
\DeclareCaptionFont{white}{\color{white}}
\DeclareCaptionFormat{listing}{%
  \parbox{\textwidth}{\colorbox{gray}{\parbox{\textwidth}{#1#2#3}}\vskip-4pt}}
\captionsetup[lstlisting]{format=listing,labelfont=white,textfont=white}
\definecolor{mygreen}{rgb}{0,0.6,0}
\definecolor{mygray}{rgb}{0.5,0.5,0.5}
\definecolor{mymauve}{rgb}{0.58,0,0.82}
\lstset{frame=lrb,xleftmargin=\fboxsep,xrightmargin=-\fboxsep}

\lstset{ %
language=Java, % choose the language of the code
basicstyle=\footnotesize, % the size of the fonts that are used for the code
backgroundcolor=\color{white}, % choose the background color. You must add \usepackage{color}
showspaces=false, % show spaces adding particular underscores
showstringspaces=false, % underline spaces within strings
showtabs=false, % show tabs within strings adding particular underscores
frame=single, % adds a frame around the code
tabsize=4, % sets default tabsize to 2 spaces
captionpos=t, % sets the caption-position to bottom
breaklines=true, % sets automatic line breaking
breakatwhitespace=false, % sets if automatic breaks should only happen at whitespace
keywordstyle=\color{blue},
stringstyle=\color{mymauve},
rulecolor=\color{black},
extendedchars=true, belowcaptionskip=3ex,
numbers=left,
escapeinside={\%*}{*)} % if you want to add a comment within your code
}

\title{OpenDataWrapper - HOW TO}
\author{Sébastien CHENAIS}
\date{juin 2013}

\begin{document}
\maketitle
\newpage

\tableofcontents %% produire à cet endroit la table des matièree					
\newpage

\section{Présentation de LodPaddle}

LodPaddle est un projet porté par Hala Skaf-Molli du LINA\footnote{Laboratoire
d'Informatique de Nantes Atlantique} et fait écho au projet Européen
Lod2\footnote{LOD2:
\href{http://lod2.eu/WikiArticle/Project.html}{http://lod2.eu/WikiArticle/Project.html}}
qui a pour but de faciliter la production et l'exploitation de données au format
spécifique du web sémantique. Pour ce faire, Lod2 nous propose une suite
d'outils qui permettent l'extraction et la conversion de donnée, leur mise en
ligne, les liens entre les différentes entités, leur enrichissement \ldots

Depuis quelques temps, le conseil général de la Loire-Atlantique, la région Pays
de la Loire et la ville de Nantes on ouvert un pole opendata dans leur
département. Actuellement composé de 422 jeux de données sur des sujets variés
comme les écoles, les loisirs, les subventions, les transports \ldots, ces
données gagneraient énormément à être sémantifiées.

Le projet LodPaddle s'est donc associé aux acteurs du site
\href{http://data.paysdelaloire.fr}{data.paysdelaloire.fr} afin de sémantifier
leurs données, de permettre aux utilisateurs et développeurs de récupérer les
données au format RDF \footnote{Ressource Description Framework} ou d'interroger
directement un \begin{itshape}SPARQL Endpoint \footnote{Serveur web comprenant
le langage de requête SPARQL}\end{itshape} et de développer une application
pilote montrant que les données sémantifiées gagnent en valeur ajoutée.

\section{Qu'est ce que le web sémantique}

Aussi appelé web 3.0 ou web des données, le web sémantique est une évolution du
web que nous connaissons. Actuellement, énormément d'information est dispersé
sur la toile, comme sur wikipédia par exemple, mais il est difficile de
récupérer la bonne information rapidement. Par exemple, vous cherchez la liste
des capitales mondiale, vous allez faire une requête sur votre moteur de
recherche préféré, chercher une page qui pourrait éventuellement proposer la
solution puis chercher dans la page choisie la probable information. Et celà est
vrai uniquement si une personne a crée cette liste au préalable.

Dans le domaine du web sémantique, la réponse serait instantanément une liste
des capitales mondiales, au format texte, exploitable par un humain et une
machine.

\begin{minipage}{\linewidth}
\begin{lstlisting}[caption=Requête SPARQL récupérant la liste des capitales
mondiales, language=SPARQL]
select distinct ?l where {
	?s prop-fr:capitale ?l.
	?s rdf:type dbpedia-owl:Country.
}
\end{lstlisting}
\end{minipage}

\end{document}
